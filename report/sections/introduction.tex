\section{Introduction}
	In a \emph{zero-sum game}, two players choose their strategies simultaneously and receive opposite payoffs. The central solution concept in zero-sum games is the \emph{value}. Under standard assumptions, a minmax theorem \cite[Appendix A.5]{Sorin2002} guarantees the existence of the value and optimal strategies, which represent the outcome of the game played by rational players.

	\emph{Zero-sum stochastic games} generalize strategic-form games to dynamic situations and Markov Decision Processes (MDPs) to competitive scenarios. Introduced by Lloyd Shapley \cite{Shapley1953}, their seminal model involves two players who repeatedly play a zero-sum game that depends on a variable called \emph{state}. The state evolves stochastically from one stage to the next, based on the players' actions and the previous state. At each stage, players know the current state and each other's past actions. For a given initial distribution and $n \in \N$, the \textit{$n$-stage game} is defined, where player 1 aims to maximize the expected average payoff over $n$ stages, and player 2 aims to minimize it. The $n$-stage game has a value, denoted by $v_n$. A key question is the convergence of $v_n$ as $n$ tends to infinity, i.e., the study of the long-duration game (see, for example, \cite{LarakiSorin2015} for a comprehensive exposition). Bewley and Kohlberg \cite{BewleyKohlberg1976} proved the existence of the \emph{limit value} for zero-sum stochastic games with finite state spaces and action sets, which can be seen as an approximate payoff solution of the game as the number of stages tends to infinity. Following this seminal work, among the long stream of works studying this question, counterexamples for infinite state spaces and/or action sets have been identified (see, e.g., \cite{Vigeral2013, Ziliotto2016}).

	Recently, the \emph{Percolation Games} model \cite{GarnierZiliotto2022} was introduced as a new model of zero-sum stochastic games with an infinite state space. In a \textit{percolation game}, two players move a token along the edges of $\mathbb{Z}^d$, with the payoff function being random in space and known in advance by the players. For a given initial state $z \in \mathbb{Z}^d$, the $n$-stage game is defined as the game where the payoff is the (random) average payoff over $n$ stages. It has been proven that for i.i.d. and \textit{oriented} percolation games, $v_n$ converges almost surely to a limit value as the duration tends to infinity. Here, ``oriented'' refers to games where the object tends to move in a fixed direction regardless of players' actions, and ``i.i.d.'' indicates that the payoff function is independent and identically distributed in space. As previously noted, studying the existence of the limit value is particularly challenging in stochastic games with infinite state spaces. Thus, the interest in this model, which despite having an infinite state space has a lot of structure.

	This model also connects with classic percolation problems, such as first passage and last passage percolation. Essentially, studying the average minimal or maximal cost to traverse a distance $n$ can be reformulated as a one-player game. This model goes one step further by adding another player. Additionally, it provides a discrete-time stochastic game framework \cite[Section 4]{GarnierZiliotto2022} suitable for studying the \textit{non-convex stochastic homogenization} of Hamilton-Jacobi equations \cite{LionsPapanicolaouVaradhan1987}. Very briefly, there is an analogy between the existence of a limit value in percolation games and the existence of an asymptotic value in continuous-time games where the environment evolves according to a differential system controlled jointly by the players, known as \textit{differential games}, which value functions are solutions to Hamilton-Jacobi equations. This connection has been explored in \cite{Ziliotto2016, FeldmanFermanianZiliotto2021}, whose authors provide examples of non-convex Hamilton-Jacobi equations that do not homogenize in a stationary ergodic environment, inspired by stochastic games with state space $\mathbb{Z}^2$ that lack a limit value.

	A natural and still open question for percolation games is whether the limit value exists without the orientation assumption. In the i.i.d. setting, the existence of the limit value is not guaranteed without this assumption, and even when it does exist, it may be random and dependent on the initial state. Studying this question presents a significant challenge, as the value can strongly depend on local properties of the environment in the absence of orientation. During this research internship, our goal was to delve a little deeper into this question by examining specific non-oriented games.
	
	First, we studied two examples in which players move a token along the vertices of the square lattice $\Z^2$, incurring the cost associated with each edge traversed, with edge costs being i.i.d. Bernoulli random variables. The main difference between them is that in the first game, the token moves only once at each stage as a result of the combined actions of both players, while in the second game, the token moves twice --once for each player's action. 
	% The first game is non-symmetric, while the second game is symmetric. 
	This part of the study is motivated by a research conducted by Ziliotto \cite{Ziliotto2023}, who, together with Avelio Sepúlveda, has worked on characterizing the critical probability of oriented bond percolation through the limit value of the first game. They proved that for this game, the limit value is independent of the initial state and deterministic, further establishing that it is 1 if and only if the Bernoulli parameter exceeds the critical probability threshold for oriented bond percolation. This motivated us to explore the conditions under which the limit value might be 0 and to investigate the second game, seeking to uncover potential connections with percolation theory.

	In analyzing these games, we focused on percolation through specific lattice structures that confer strategic advantages to the players. These structures, along with their associated critical probabilities, are crucial for devising optimal gameplay strategies. For the first game, we provided a geometric characterization of the advantageous structure for player 1, corresponding to a value of zero (since player 1 aims to minimize), and demonstrated the non-triviality of its critical probability. For the second game, we employed a similar approach and attempt to prove that its limit value is also independent of the initial state and deterministic. Additionally, we established a geometric characterization of the advantageous structures for each player and showed that their critical probabilities are non-trivial. We formulated several conjectures that, while strongly believed to be true, we were unable to prove conclusively. To test these conjectures and gain further insights, we conducted computational simulations for both games. This approach was particularly valuable for the second game, where simulations significantly enhanced our understanding. In this game, the distribution of 0s and 1s across the structures emerged as a significant factor.

	In the second part of our work, we extended the original i.i.d. and oriented percolation games by introducing a stochastic process into the first component of the game's state transitions. This modification allows the token to return to previously visited states while remaining oriented in expectation, provided the stochastic process has a mean of zero. We then analyzed the value of this new game and proved that if the stochastic process also has bounded support, the resulting game possesses a limit value and that its expected value has a rate of convergence which, up to a constant factor, is equal to that of the original percolation game.

	In the following section, we provide a formal description of the Percolation Games model and summarize all relevant aspects and known results. In Section 3, we present the two examples and the theoretical analysis conducted, followed by a description of the computational methods used and a discussion of the results obtained. In Section 4, we introduce the generalized model mentioned above and prove our main result. Finally, in Section 5 we conclude and give our perspectives for future work.


	\subsubsection*{Notations} The set of nonnegative integers is denoted by $\N$, and by $\N_+ := \N \setminus \{0\}$. The set of real numbers is denoted by $\R$ and by $\bar{\R} := \R \cup \{+\infty, -\infty\}$.

	The set of integers is denoted by $\Z$ and by $\Z^d := \{(x_1, x_2, \ldots, x_d) : x_i \in \Z, \text{for } i =  1, 2, \ldots, d\}$. We turn $\Z^d$ into a graph by adding an (undirected) edge between every pair of vectors $x$ and $y$ such that $|x - y| = 1$. The resulting set of edges is denoted by $\E^d$. In this context, we refer to the vectors in $\Z^d$ as vertices. The origin of this graph is the vertex $0 = (0, 0, \ldots, 0)$. 

	The square lattice rotated $45^\circ$ is denoted by $\Z^2_{r}$, and its edge set by $\E_r^2$.

	For $x, y \in \Z^d$, we write $x \leq y$ if $x_i \leq y_i$ for $1 \leq i \leq d$, and we let $\vec{\mathbb{L}}^d = (\Z^d, \vec{\mathbb{E}}^d)$ denote the directed graph obtained by assigning an arrow from $x$ to $y$ whenever $\{x, y\} \in \E^d$ and $x \leq y$.

	For $a, b \in \Z$, we denote $[a..b] := \{i \in \Z : a \leq i \leq b\}$, and for $n \in \N_+$, $[n] := [1..n]$.

	For a vector $\mathbf{v} = (v_1, v_2, \ldots, v_d) \in \R^d$, its $k$-th component is denoted as $[\mathbf{v}]_k := v_k$, and $\|\mathbf{v}\|$ represents its Euclidean norm.

	% For vectors $\mathbf{a}$ and $\mathbf{b}$ in $\R^d$, the vector projection of $\mathbf{a}$ onto  $\mathbf{b}$ is denoted as $\vectproj[\mathbf{b}]{\mathbf{a}}$.

	Let $f, g$ be real functions. If $f(x)$ is big-O of $g(x)$ as $x$ tends to $\infty$, we write $f(x) = O(g(x))$.

	Given a graph $G =(V, E)$, a \emph{percolation configuration} $\omega = (\omega_e \colon e \in E)$ is an element of $\{0, 1\}^E$. If $\omega_e = 1$, the edge $e$ is said to be \emph{open}; otherwise, $e$ is said to be \emph{closed}. A \emph{percolation model} is defined by a distribution on percolation configurations on $G$. 

	The simplest example of a percolation model is \emph{Bernoulli percolation} (also known as bond percolation), in which each edge is open with probability $p$ and closed with probability $1-p$, independently of the states of other edges. Another model is \emph{site percolation} in which the sites (or vertices) of the lattice are open or not, with probability $p$ and $1 - p$, respectively.

	We will consider Bernoulli percolation on the lattice $\Z^d$, $\Z_r^d$ and $\vec{\mathbb{L}}^d$, the latter of which is known as \emph{oriented bond percolation}. For each graph, with it corresponding configuration space $\{0, 1\}^E$, we associate the $\sigma$-algebra $\F$ generated by events depending on finitely many edges. For $0 \leq p \leq 1$, we let $\Pbb_p$ denote the corresponding product measure on $(\{0, 1\}^E, \F)$ with density $p$. 