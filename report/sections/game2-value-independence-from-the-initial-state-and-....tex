The limit value of this game, like that of Game 1, is expected to be independent of the initial state and deterministic. However, in this case, we have only been able to prove that its value depends on whether the sum of the coordinates of the initial state $z$ is even or odd. To establish this result, we follow the approach outlined by Ziliotto in \cite{Ziliotto2023}, which proves a corresponding result for Game 1, as stated in Theorem \ref{theorem-vz-independent-z-game1}.

\begin{theorem}
      Let $V_{even} := \{(x, y) \mid x+y \text{ is even}\}$ and $V_{odd} := \{(x, y) \mid x+y \text{ is odd}\}$. Then, $v_p(z) = v_p(z')$ for all $z, z' \in V_{even}$, and $v_p(z) = v_p(z')$ for all $z, z' \in V_{odd}$.  
\end{theorem}

\begin{proof}
    Let $z, z' \in V_{even}$. Assume $z'$ is on the right of $z$. 

    Let player 1 always play top and player 2 play optimally from $z$. Denote the path induced by these strategies as $\pi_{top}$. Since player 2 has only two options, left or right, $\pi_{top}$ will be a vertically oriented path from $z$ upwards.

    We claim that,
    \begin{equation}\label{inequality-v-top}
        v_p(z'') \leq v_p(z), \text{ for all $z'' \in \pi_{top} \cap V_{even}$.}
    \end{equation} 

    Indeed, let $\tau$ denote the strategy of player 2 playing optimally from $z$, and let $\sigma''$ be the strategy of player 1: play top until $z''$, then play optimally from $z''$.

    Since we are considering the limit superior criterion for aggregating the payoffs, which does not depend on any finite number of stages, all the payoffs between $z$ and $z''$ are negligible for \eqref{payoff-game1}. More formally, if $z$ and $z''$ are at distance $M$ in $\pi_{top}$, we have
    \begin{equation*}%\label{equality-gammas-z-z''}
        \gamma(z, \sigma'', \tau) = \limsup_{n \to \infty}\frac{1}{2n}\sum_{m = 1}^{2n}c(e_m) = \limsup_{n \to \infty}\frac{1}{2n}\sum_{m = M}^{2n}c(e_m) = \gamma(z'', \sigma'', \tau).
    \end{equation*}
    Because $\tau$ is optimal from $z$ and player 2 is minimizing, $v_p(z) \geq \gamma(z, \sigma'', \tau)$. Additionally, since $\sigma''$ is optimal from $z''$ and player 1 is maximizing, $\gamma(z'', \sigma'', \tau) \geq v_p(z'')$. Thus, we obtain the claimed inequality. 

    Now, assume player 1 always plays bottom and player 2 plays optimally from $z$. This time, the path $\pi_{bottom}$ will be a vertically oriented path from $z$ downwards. By following the same reasoning, we obtain that,
    \begin{equation}\label{inequality-v-bottom}
        v_p(z'') \leq v_p(z), \text{ for all $z'' \in \pi_{bottom} \cap V_{even}$.}
    \end{equation} 
    Thus, advancing through $\pi := \pi_{top} \cup \pi_{bottom}$, and more precisely through the vertices of $\pi \cap V_{even}$, in any direction only decreases the value.

    To relate $v_p(z)$ to $v_p(z')$, consider moving from $z'$ towards $z$. Start from $z'$, with player 2 always playing left and player 1 playing optimally from $z'$. Denote the path induced by these strategies as $\pi_{left}$. Since player 1 has only two options, top or bottom, there are 2 possibilities: $\pi_{left}$ either eventually crosses $\pi$ or it does not. For both cases, we claim that:
    \begin{equation} \label{inequality-zprime-zhat-z}
          v_p(z') \leq v_p(z).
    \end{equation} 
    From this, and by symmetry between $z$ and $z'$, we can conclude the desired equality: $v_p(z') = v_p(z)$ for all $z, z' \in V_{even}$.  

    Let us first analyze the case where the paths intersect. Let $\hat{z} \in \pi_{left} \cap \pi$ be the first intersection point. Note that $\hat{z} \in V_{even}$. Thus, if $\hat{z}$ is located at the top (or bottom) of $z$, by \eqref{inequality-v-top} (or \eqref{inequality-v-bottom}), we have
    \begin{equation*}
        v_p(\hat{z}) \leq v_p(z).
    \end{equation*}
    Furthermore, following the same line of reasoning as before, when proving \eqref{inequality-v-top}, but reversing players role, we obtain 
    \begin{equation*}
         v_p(\hat{z}) \geq v_p(z'). 
    \end{equation*}
    
    If $\pi_{left} \cap \pi = \emptyset$, then the only possibility is that both paths ``go in the same direction''. Therefore, $\pi_{left}$ must have either a finite number of top or bottom turns, and $\pi$ must have a finite number of right turns. Without loss of generality, assume $\pi_{left}$ contains a finite number of tops, then $v_p(z') \leq p$, and from $\pi$ we obtain that $v_p(z) \geq p$. 

    Finally, a similar approach can be applied to $z, z' \in V_{odd}$.

\end{proof}

\begin{remark} 
    This theorem establishes that $v_p$ remains the same regardless of which vertex in $V_{even}$ or $V_{odd}$ the game starts from. However, it may differ depending on whether the starting vertex belongs to $V_{even}$ or $V_{odd}$. We initially conjectured that this result could be extended to all $z \in \Z^2$, asserting that $v_p(z)$ is independent of the initial state $z \in \Z^2$. Although we initially believed we had proven this, errors were identified in the proof shortly before submission. We were unable to resolve these issues and decided to include this remark for clarity.
\end{remark}

If the value were independent on the initial state, the event $\{v_p > x\}$ would be invariant under translations (see the definition of the Percolation Games model in Section 2.1). By the ergodicity of $\Pbb_p$ (see Lemma 2.8 in \cite{DuminilCopin2018}), it would follow that $\Pbb_p(v_p > x) \in \{0, 1\}$ for all $x \in \R$, implying the existence of $c \in \R$ such that $v_p =  c$ almost surely.

Since we will continue working together during my PhD, we aim to complete the proof as soon as we begin. We believe the proof can be established by carefully considering whether it is Player 1's or Player 2's turn. A key question is whether the value when Player 1 starts is the same as when Player 2 starts.