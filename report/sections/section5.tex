\section{Conclusions and Perspectives}
In Section 3, we saw how studying the limit value in two non-oriented percolation games on the square lattice is equivalent to analyzing the emergence of certain structures within the percolation model on the same lattice. This connection is particularly intriguing, though not entirely unexpected, as percolation games can be seen as a dynamic, strategic extension of percolation theory, where players actively influence or look for the percolation process rather than passively observing random phenomena.

Percolation is one of the simplest and most well-known examples of phase transition, yet many aspects remain unresolved (for example, despite extensive research, no exact solution has been found for the site percolation problem on the square lattice), which has led to wide use of numerical simulations in this field. We believe that exploring the connection with percolation games could provide valuable insights for further studies.

Although the examples we considered are highly specific, they lay a foundation for understanding more complex systems and may offer valuable insights into the nature of the limit value without the orientation assumption.

In Section 4, by proving Theorem \ref{theorem-main}, we demonstrated that the results obtained by Garnier and Ziliotto in \cite{GarnierZiliotto2022} for i.i.d. and oriented percolation games remain valid even when certain stochastic elements are introduced in state transitions, provided these transitions maintain the game's orientation in expectation. 

This work directly contributes to my PhD, which will be co-supervised by Bruno Ziliotto, Guillaume Vigeral, and Laurent Miclo. The primary goal of the PhD is to study percolation games in depth, with a particular focus on the existence of the limit value without the orientation assumption. One of the project's initial tasks will be to extend the results of Section 4 to a ``continuous space and time'' context within a differential game framework. As noted in the introduction, this approach is particularly relevant for addressing the problem of stochastic homogenization of non-convex Hamilton-Jacobi equations. Additionally, we aim to generalize this result as much as possible by considering random fluctuations in multiple components and extending to a $d$-dimensional stochastic process. We also intend to address the unresolved issues and conjectures proposed in Section 3 and continue exploring new connections with percolation.

% As noted in the introduction and at the beginning of that section, this approach is particularly relevant to address the problem of stochastic homogenization of non-convex Hamilton-Jacobi equations. In addition, we intend to address all the conjectures proposed in Section 3 and to continue searching for and studying new connections with percolation.