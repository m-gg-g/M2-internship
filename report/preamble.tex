\usepackage[right = 3.0cm, top = 3.5cm, bottom = 3.5cm, left = 3.0cm]{geometry}

\usepackage[english]{babel}
\usepackage[T1]{fontenc} % este y el de abajo son importantes para las tildes en acentuación

\usepackage{float} % for floating environments
\usepackage{stmaryrd, amsfonts, amssymb, amsmath, amsthm, mathrsfs, mathtools} % for mathematical symbols and equations, and other things
\usepackage{epsfig, graphicx, subfigure} % for inserting figures
\usepackage{caption, subcaption} % for adjusting captions of tables and figures
\usepackage{color} % for writing colored texts
\usepackage{setspace} % for adjusting line spacing
\usepackage{titlesec} % for theorem´s enumeration, etc.
% \usepackage{bbm} % for using \mathbbm{1}
\usepackage{url} % tú que crees
\usepackage{bbm} % para usar \mathbbm{1}
\usepackage{comment} % for comment text

\usepackage{makeidx} % no sé, no me acuerdo

% \usepackage{mdframed}
% % Define theorem environment inside a box
% % Define boxed theorem environment
% \newmdtheoremenv[
%   linecolor=black,
%   linewidth=0.5pt,
%   backgroundcolor=white,
%   innertopmargin=5pt,
%   innerbottommargin=7pt
% ]{theorem}{Theorem}[section]

% \newmdtheoremenv[
%   linecolor=black,
%   linewidth=0.5pt,
%   backgroundcolor=white,
%   innertopmargin=5pt,
%   innerbottommargin=7pt
% ]{proposition}{Proposition}[section]

% \newmdtheoremenv[
%   linecolor=black,
%   linewidth=0.5pt,
%   backgroundcolor=white,
%   innertopmargin=5pt,
%   innerbottommargin=7pt
% ]{lemma}{Lemma}[section]

% \newmdtheoremenv[
%   linecolor=black,
%   linewidth=0.5pt,
%   backgroundcolor=white,
%   innertopmargin=5pt,
%   innerbottommargin=7pt
% ]{conjecture}{Conjecture}[section]

% redefiniéndolos para que se enumeren según las secciones
\theoremstyle{plain} % in italic
\newtheorem{theorem}{Theorem}[section]
\newtheorem{lemma}[theorem]{Lemma}
\newtheorem{proposition}[theorem]{Proposition}
\newtheorem{conjecture}[theorem]{Conjecture}
\theoremstyle{definition} % normal
\newtheorem{definition}[theorem]{Definition}
\newtheorem{remark}[theorem]{Remark}

\def\<#1>{\mathinner{\langle#1\rangle}}

\usepackage{tikz}
\usetikzlibrary{graphs, graphs.standard, quotes, shapes.geometric}
\usepackage{ifthen}

\tikzset{costLine/.style={ultra thin, line width=0.5pt, color=blue}}

\newcommand{\costOne}{
    \begin{tikzpicture}[baseline, yshift=0.5ex]
        \draw[costLine] (0, 0) -- (1, 0);
    \end{tikzpicture}
}

\newcommand{\costZero}{
    \begin{tikzpicture}[baseline, yshift=0.5ex]
        \draw[dashed, costLine] (0, 0) -- (1, 0);
    \end{tikzpicture}
}
\definecolor{blue(pigment)}{rgb}{0.2, 0.2, 0.6}

\newcommand\norm[1]{\left\lVert#1\right\rVert}

% para el paseudocódigo
\usepackage{algorithm}
\usepackage{algorithmic}

% para que los hyperlinks no tengan color
\usepackage[colorlinks=false]{hyperref} % links

% for the E of the probability space, \mathcal{E}
% \usepackage{boondox-cal}

% redefining some math commands
\newcommand{\R}{\mathbb{R}}
\newcommand{\Z}{\mathbb{Z}}
\newcommand{\N}{\mathbb{N}}
\newcommand{\C}{\mathbb{C}}
\newcommand{\E}{\mathbb{E}}
\newcommand{\Pbb}{\mathbb{P}}
\newcommand{\F}{\mathscr{F}}
\newcommand{\Lip}{\text{Lip}}

% command for the projection of a vector 1 onto a vector 2
\DeclareMathOperator{\proj}{proj}
\newcommand{\vct}{\mathit}
\newcommand{\vectproj}[2][]{\proj_{\vct{#1}}[\vct{#2}]}

% for the fill square in the end of the proofs
\renewcommand{\qedsymbol}{\ensuremath{\blacksquare}}
\renewenvironment{proof}{{\bfseries Proof.}}{\qed}

\newenvironment{someideasfortheproof}{%
  {\bfseries Some ideas for the proof.}%
}{%
  \qed
}


% for the special arrows in the description of the examples
\makeatletter
\newcommand{\fixed@sra}{$\vrule height 2\fontdimen22\textfont2 width 0pt\shortrightarrow$}
\newcommand{\shortarrow}[1]{%
  \mathrel{\text{\rotatebox[origin=c]{\numexpr#1*45}{\fixed@sra}}}
}
\makeatother

% spacing between parragraphs and before its start
%\parindent = 0mm % no deja sangría
\setlength{\parindent}{1em} % deja sangría
\setlength{\parskip}{6pt} % Espacio entre párrafos
