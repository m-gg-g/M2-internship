\begin{figure}[h]
    \centering
    \begin{tikzpicture}[scale=0.5pt]
    \def\a{-5.5}
    \def\b{10.5}

    \draw[color = black!100] (0, 2) -- (3, 2);
    \draw[color = black!100] (0, 3) -- (3, 3);

    \foreach \x in {0, ..., 3} {
        \draw[color = black!100] (\x, 2) -- (\x, 2+1);
    }

    \draw[color = black!100] (2, 1) -- (10, 1);
    \draw[color = black!100] (2, 2) -- (10, 2);

    \foreach \x in {2, ..., 10} {
        \draw[color = black!100] (\x, 1) -- (\x, 1+1);
    }

    \draw[color = black!100] (9, 2) -- (12, 2);
    \draw[color = black!100] (9, 3) -- (12, 3);

    \foreach \x in {9, ..., 12} {
        \draw[color = black!100] (\x, 2) -- (\x, 2+1);
    }

    \draw[color = black!100] (11, 3) -- (15, 3);
    \draw[color = black!100] (11, 4) -- (15, 4);

     \foreach \x in {11, ..., 15} {
        \draw[color = black!100] (\x, 3) -- (\x, 3+1);
    }

    \draw[color = black!100] (14, 4) -- (16, 4);
    \draw[color = black!100] (14, 5) -- (16, 5);

     \foreach \x in {14, ..., 16} {
        \draw[color = black!100] (\x, 4) -- (\x, 4+1);
    }

    \draw[color = black!100] (15, 5) -- (21, 5);
    \draw[color = black!100] (15, 6) -- (21, 6); 
    \foreach \x in {15, ..., 21} {
        \draw[color = black!100] (\x, 5) -- (\x, 5+1);
    }

    \draw[color = black!100] (20, 4) -- (23, 4);
    \draw[color = black!100] (20, 5) -- (23, 5); 
    \foreach \x in {20, ..., 23} {
        \draw[color = black!100] (\x, 4) -- (\x, 4+1);
    }

    \draw[color = black!100] (22, 5) -- (24, 5);
    \draw[color = black!100] (22, 6) -- (24, 6);
    \foreach \x in {22, ..., 24} {
        \draw[color = black!100] (\x, 5) -- (\x, 5+1);
    }
    \end{tikzpicture}
    \caption{Possible portion of a winning structure for player 1 in Game 2.}
    \label{fig_portion_winning_structure_player1_game2}
\end{figure}